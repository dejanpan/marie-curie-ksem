\section{RESEARCHER} % (MAXIMUM 7 PAGES WHICH INCLUDES A CV AND A LIST OF MAIN ACHIEVEMENTS) 
\label{sec:researcher}
\subsection{Research experience}
Mr. Pangercic, the applicant of the \ksem\ project, received his B.Sc. degree
with distinction from the University of Ljubljana in 2003. In his thesis he investigated
a Wishbone Bus and its applicability to let the parts of an integrated circuit communicate with each other.
In 2004 he joined the Laboratory for Integrated Circuit Design, Faculty of Electrical Engineering
Ljubljana for 3 months to work on the system for Power and Data Transmission over one Line.
His system was later on commercialized and deployed to the gas warehouse in west Slovenia.
In 2005 Mr. Pangercic enrolled to the Technische Universit\"at M\"unchen (TUM) in to pursue Master's
degree. In 2007 he successfully defended his Master's Thesis titled Monocular 3D SLAM for Indoor Environments
which he carried out under the supervision of Prof. Michael Beetz of Intelligent Autonomous Systems Group.
The thesis generated his first publication at the major international conference, namely
Emerging Technologies on Factory Automation. Upon this he was accepted to the graduate school
of the Computer Science Department of TUM where he is projected to graduate in early 2012.
In his PhD studies Mr. Pangercic deepened the understanding and investigated the connection of the 
perception algorithms for personal robotics with the knowledge representation for robots
acting in household environments. Hence the title of his thesis: Knowledge-enabled Scene Perception for Personal Robotics.
During this time he published at the prime robotics conferences and four times appeared as first author, 
and contributed to seven publications as secondary author. He also co-authored two journal publications listed below as well.
All publications are listed on pages \todo{XX} and \todo{YY}.

Mr. Pangercic actively participates in community activities and attends all major conferences, for 
which he also serves as a reviewer (ROMAN, ETFA, ICRA, IJRR, IROS, ICAR, CVPR, IJCAI). In 2009 he 
attended the Robot Learning Summer School where he gained a substantial knowledge on machine learning 
applied to the practical robotics problems. On the other hand he co-organized Player Summer School on 
Cognitive Robotics~\footnote{http://www9.cs.tum.edu/events/psscr07/} and 
served as a main organizer for the CoTeSys-ROS Fall School on Cognition-enabled Mobile 
Manipulation~\footnote{http://www.ros.org/wiki/Events/CoTeSys-ROS-School} which hosted 
65 researchers from Europe and USA. He also, together with J\"urgen Sturm, organized a highly 
attended RGB-D Workshop on 3D Perception in Robotics~\footnote{http://ias.cs.tum.edu/events/rgbd2011} 
that took place the European Robotics Forum 2011.

Since May 2011 he is with Bosch Research and Technology Center in Palo Alto where he is working 
on the 3D Environment Reconstruction and Semantic Mapping system that will serve as a basis for
the herein proposed work. The part of his work at Bosch will be demonstrated at the IROS 2011
conference which will take place between September 25th and 29th. \\ 
% Writen CV: Uni Ljubljana, TUM, Bosch, Willow, 
% Major Projects: Power and Data over one line, VSLAM, combination of perception and knowledge, 3D perception 
% and modelling at Bosch, ROS school \\
\subsection{Research results including patents, publications, teaching etc., taking into account the level 
of experience} publicatons, supervision of students, ROS consulting, ROS training, crosos proposal, pr2 proposal\\
\subsection{Independent thinking and leadership qualities} 
Mention postdoc role in Michael's group, mention following ideas: shopping demo, sift+vocabulary trees, pancake demo, 
leading in terms of meeting, LAAS workshop, recruiting of new people\\
\subsection{Match between the fellow's profile and project}
mention semantic mapping paper, AAAI video, shopping demo, CopMan\\
\subsection{Potential for reaching a position of professional maturity}
Mr. Pangercic is a skillful researcher, who has already had the chance to integrate in a highly 
competitive scientific environment other than the university at which he is conducting his
studies.  His participation in a European program like \ksem will enable him to expand 
his scientific knowledge and skills. Expanding his work in such a high-standard research 
and educational institutions, will enable her to further increase her 
research experience and will develop her collaboration skills, integrate and  provide his 
ways to establish himself in the wide European scientific community. The gain of research 
experience in a competitive environment will be the step forward that will soon help him
attain a position of professional maturity in a university or a research institution. The 
measures foreseen to help the applicant reach professional maturity are connected to the 
possibility of pursuing a novel challenging research project in a well established research 
institution and a group which has successfully trained junior scientists in the past. 
Specifically, within two more years of postdoctoral experience, new knowledge will be 
acquired together with new scientific and communicative/collaborative skills, additional 
teaching experience will be gained, as well as tutoring of graduate and undergraduate 
students in the partner group. \\
\subsection{Potential to acquire new knowledge}
manipulation of deformable objects, reinforcement learning,  american business model, start-ups, 
american educational model\\
\includepdf[pages=1-3]{pangercic_cv.pdf}
\newpage
