\section{RESEARCHER} % (MAXIMUM 7 PAGES WHICH INCLUDES A CV AND A LIST OF MAIN ACHIEVEMENTS) 
\label{sec:researcher}
\subsection{Research experience}
Mr. Pangercic, the applicant of the \ksem\ project, received his B.Sc. degree
with distinction from the University of Ljubljana in 2003. In his thesis he investigated
a Wishbone Bus and its applicability to let the parts of an integrated circuit communicate with each other.
In 2004 he joined the Laboratory for Integrated Circuit Design, Faculty of Electrical Engineering
Ljubljana for 3 months to work on the system for Power and Data Transmission over one Line.
His system was later on commercialized and deployed to the gas warehouse in west Slovenia.
In 2005 Mr. Pangercic enrolled to the Technische Universit\"at M\"unchen (TUM) in to pursue Master's
degree. In 2007 he successfully defended his Master's Thesis titled Monocular 3D SLAM for Indoor Environments
which he carried out under the supervision of Prof. Michael Beetz of Intelligent Autonomous Systems Group.
The thesis generated his first publication at the major international conference, namely
Emerging Technologies on Factory Automation. Upon this he was accepted to the graduate school
of the Computer Science Department of TUM where he is projected to graduate in early 2012.
In his PhD studies Mr. Pangercic deepened the understanding and investigated the connection of the 
perception algorithms for personal robotics with the knowledge representation for robots
acting in household environments. Hence the title of his thesis: Knowledge-enabled Scene Perception for Personal Robotics.
During this time he published at the prime robotics conferences and four times appeared as first author, 
and contributed to seven publications as secondary author. He also co-authored two journal publications listed below as well.
All publications are listed on pages \todo{XX} and \todo{YY}.

Mr. Pangercic actively participates in community activities and attends all major conferences, for 
which he also serves as a reviewer (ROMAN, ETFA, ICRA, IJRR, IROS, ICAR, CVPR, IJCAI). In 2009 he 
attended the Robot Learning Summer School where he gained a substantial knowledge on machine learning 
applied to the practical robotics problems. On the other hand he co-organized Player Summer School on 
Cognitive Robotics~\footnote{http://www9.cs.tum.edu/events/psscr07/} and 
served as a main organizer for the CoTeSys-ROS Fall School on Cognition-enabled Mobile 
Manipulation~\footnote{http://www.ros.org/wiki/Events/CoTeSys-ROS-School} which hosted 
65 researchers from Europe and USA. He also, together with J\"urgen Sturm, organized a highly 
attended RGB-D Workshop on 3D Perception in Robotics~\footnote{http://ias.cs.tum.edu/events/rgbd2011} 
that took place the European Robotics Forum 2011.

Since May 2011 he is with Bosch Research and Technology Center in Palo Alto where he is working 
on the 3D Environment Reconstruction and Semantic Mapping system that will serve as a basis for
the herein proposed work. The part of his work at Bosch will be demonstrated at the IROS 2011
conference which will take place between September 25th and 29th. \\ 
% Writen CV: Uni Ljubljana, TUM, Bosch, Willow, 
% Major Projects: Power and Data over one line, VSLAM, combination of perception and knowledge, 3D perception 
% and modelling at Bosch, ROS school \\
\subsection{Research results including patents, publications, teaching etc., taking into account the level 
of experience} 
The main research results of his PhD work at TUM are presented in publications, in
peer-reviewed journals and his presentations in conferences and seminars, showing his 
potential in highly applied research. In most of these publications, Mr. Pangercic is the  first 
author and primary researcher. Among the results are included: i) Global Radius-based Surface
Descriptor which is a 3D feature useful to perform categorization of objects, ii) \textsc{K-CoPMan}
(Knowledge-enabled Cognitive Perception for Manipulation) which is a relization of a system that 
combines robotic perception algorithms with symbolic representations of thereof and thus enable
e.g. life long learning of semantic object maps and ii) realization of
a system for static semantic mapping of environments which will be of a high relevance to \ksem\ project.
The results exhibit the integration of diverse skills of the applicant such as the ability  to tackle 
a problem and apply the most efficient computational scheme to resolve it.

Mr. Pangercic has been involved into teaching and supervision of undergraduate 
student projects ever since he joined the PhD program at TUM. The list of classes 
that he has led alone include a seminar on Intelligent Systems (WS2010), a practical 
course on Sensor-enabled Intelligent Environments and a practical course on the 
Introduction to Computer Science (WS2008). Additionally he has supervised 9 Bachelor
or Master's theses as outlined in his CV below. Most of his students have graduated
with distinction and their works have resulted in the publications at the international
conferences. Furthermore, Rok Tavcar, Andreas Leha and Martin Schuster all moved on 
to their PhD studies in the field and remain in close collaboration with Mr. Pangercic.
Several other students managed to get high-profile jobs (e.g. BMW) in world renowned 
companies. 

Mr. Pangercic also served as a main organizer for the previously mentioned CoTeSys-ROS Fall School on 
Cognition-enabled Mobile Robotics where he among other matters was in charge for the technical
program and selection of speakers. He also actively participates in discussions and roadmaps
pertinent to the development of ROS and gives tutorials and educates researchers about
ROS e.g.~\footnote{\url{http://ias.cs.tum.edu/teaching/ros-school-2011}}. He is also a part of
the expanded list of developers for Point Cloud Library~\footnote{\url{pointclouds.org}}.
 
%publicatons, supervision of students, ROS consulting, ROS training, crosos proposal, pr2 proposal\\
\subsection{Independent thinking and leadership qualities} 
During his stay at TUM and after gaining experience in the field, the applicant researcher 
showed his independent thinking when he steered the goals of his research to the
modelling of a system that will integrate the knowledge for robots representation concepts
with the low-level concepts such as the ones from the perception for robots.
His was one of the main architects behind the two demonstrations where IAS researchers
tought their robots to prepare two simple meals: baking of pancakes and cooking of 
saussages that generated gross interest among the research as well as the public
society. Both events are briefly summed up in the following 
article:\\~\begin{small}\url{http://ias.cs.tum.edu/news/robotic-roommates-shopping-for-and-preparing-bavarian-breakfast}\end{small}.

Lack of the postdoc fellows in the IAS group forced Mr. Pangercic to take on senior research
jobs such above mentioned supervising of undergradute students and writing of research grants.
For TUM he thus lead the writing of the PR2 Beta Program proposal titled CRAM (Cognitive Robot Abstract Machine)
with which TUM received a \$400.000,00 worth robotic platform PR2 and a participation in a highly 
prestigious PR2 Beta Program that involves institutes such as Stanford, Berkeley, MIT, UCS, etc.
Mr. Pangercic also contributed to the writing of a Large-scale integrating project proposal titled Web-enabled and Experience-based 
Cognitive Robots that Learn Complex Everyday Manipulation Tasks and was successfully awarded to 
the Principal Investigator Michael Beetz. Proposal's budget is estimated to \euro7.000.000,00.

Applicant's wide involvment into a research society and his recognition led to multiple collaborations
between IAS group and researchers from other institutions. In 2010 Mr. Pangercic
thus worked on the development of an RGB-D feature called VOSCH together with
Asako Kanezaki~\footnote{\url{http://www.ros.org/wiki/vosch}} from University of Tokio.
With J\"urgen Sturm from University of Freiburg he organized an earlier mentioned
RGBD workshop and at the time of writting of this proposal he is staying with Bosch RTC
in Palo Alto working on the environment reconstruction and semantic annotation
of scenes using Mechanical Turk interface and WUP similarities.
% Mention postdoc role in Michael's group, mention following ideas: shopping demo, sift+vocabulary trees, pancake demo, 
% leading in terms of meeting, LAAS workshop, recruiting of new people\\
\subsection{Match between the fellow's profile and project}
%mention semantic mapping paper, AAAI video, shopping demo, CopMan\\
The project that we propose matches perfectly Mr. Pangercic's scientific profile as it combines
his past research work in both robotic perception and knowledge representation for robots
during his graduate years at TUM. His work on detection and recognition of everyday objects as
described in \emph{Multimodal Perception System for Novel Object Modeling and Re-detection}
and work on acquisition of static maps as proposed in \emph{Autonomous Semantic Mapping for Robots Performing  
Everyday Manipulation Tasks in Kitchen Environments} will constitute a solid base for the
implementation of work proposed in WP1 above. Furthermore, his K-CoPMan system as descriped im
\emph{Combining Perception and Knowledge Processing for Everyday Manipulation} paper will
provide foundations for the second and third task in WP3. His collaborative work with Martin Schuster
and Dominik Jain on WUP similarities will enable an advancement of research on perception
of scenes as proposed in task two of WP2.

Last but not least, his excellent network of fellow researchers, knowledge of implemented algorithms
in ROS and commitment to assistance from other lead scientists from PR2 Beta Program make
matching between Mr. Pangercic profile and the project very promising.

\subsection{Potential for reaching a position of professional maturity}
Mr. Pangercic is a skillful researcher, who has already had the chance to integrate in a highly 
competitive scientific environment other than the university at which he is conducting his
studies.  His participation in a European program like \ksem will enable him to expand 
his scientific knowledge and skills. Expanding his work in such a high-standard research 
and educational institutions, will enable her to further increase her 
research experience and will develop her collaboration skills, integrate and  provide his 
ways to establish himself in the wide European scientific community. The gain of research 
experience in a competitive environment will be the step forward that will soon help him
attain a position of professional maturity in a university or a research institution. The 
measures foreseen to help the applicant reach professional maturity are connected to the 
possibility of pursuing a novel challenging research project in a well established research 
institution and a group which has successfully trained junior scientists in the past. 
Specifically, within two more years of postdoctoral experience, new knowledge will be 
acquired together with new scientific and communicative/collaborative skills, additional 
teaching experience will be gained, as well as tutoring of graduate and undergraduate 
students in the partner group. \\
\subsection{Potential to acquire new knowledge}
The applicant researcher has multiply demonstrated in the past the ability to acquire new 
knowledge after switching to a different research field and a new group. Although his
studies were in the electrical engineering field, he very 
easily integrated and undertook a project in a new group conducting  research in personal
robotics. The researchers ability to acquire new knowledge was evident in terms of 
the different scientific fields and application domains he could easily adopt himself (e.g. 3D perception, machine
learning and first-order logic). Assets, which show the high potential of the applicant to acquire new 
knowledge and use this knowledge for conducting high level research, modeling novel 
systems and guiding/proposing novel applications. In this respect, at the 
outgoing host institution, he will also have the potential to acquire new knowledge and skills. 
Specifically, he will have the possibility to be exposed to new tools and methods, such as 
apprentice learning schema, perception of deformable objects, 
Potential collaborations and interaction with the many excelling 
robotic groups in the area around Berkeley will also 
expand the applicants scientific horizons. 
%manipulation of deformable objects, reinforcement learning,  american business model, start-ups, 
%american educational model\\
\includepdf[pages=1-3]{pangercic_cv.pdf}
\newpage
