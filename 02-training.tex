\section{TRAINING} %2 pages
\label{sec:training}
\subsection{Clarity and quality of the research training objectives for the researcher}
During his doctoral thesis, Mr. Pangercic became an expert in 3D perception for personal robots, 
semantic mapping  and knowledge representation systems for personal robots. His move to a new 
research setting in the group  of Bosch RTC in Palo Alto where he is researching at the moment has 
given him the opportunity to work on environment reconstruction using meshing and texture re-projection 
techniques in order to make the robot world model look realistic. Furthermore, his integrated work with 
Dr. Ben Pitzer from Bosch and Dr. J\"urgen Sturm from TUM also enabled him to implement 
a robust system pipeline where the robot autonomously constructs the representation of 
of the world and successfully makes use of it in later stages. This work will provide a solid
foundation for the goals proposed in WP2~\ref{sec:wp1} of \ksem\ project.

Applicant's excellent integration and connection with the research staff from the return
host organization will provide expertise needed to achieve the research objectives proposed
in WP1~\ref{sec:wp1}. In particular a collaboration with Moritz Tenorth and Lars Kunze, both
experts in knowledge representation and reasoning for personal robots guarantees a successful
end of that subtask as well.

Of particular interest of Mr. Pangercic is the realization of the WP3~\ref{sec:wp3} which
will happen in conjunction with both supervisors at Berkeley, Pieter Abbeel and Ken Goldberg.
Prof. Abbeel's outstanding knowledge about machine learning (reinforcement and apprenticeship
learning in particular) will enable the applicant to deepen his knowledge in the area he has
so far not had chance to become an expert in. Prof. Goldberg's expertise in LQG-based motion planing 
will enable the applicant to make use of e.g. Gaussian models of uncertainty in order to asses
the quality of e.g. robot arm motion path.

Furthermore, it is expected that an excellent selection of graduate courses, invited talks and 
seminars~\footnote{\url{http://goldberg.berkeley.edu/index.html\#A5, 
 http://inst.eecs.berkeley.edu/~cs188/sp11/announcements.html}} at EECS Department at Berkeley and 
personal discussions with Prof. Abbeel and Prof. Goldberg will further improve applicant's research 
horizon.
% \begin{itemize}
% \item precise title for lecture, seminars, discussion, regular meetings and reading group
% \item talk about the themes and divide this into outgoing and return host
% \item Grant enables me to come from A to B to C
% \item How have I integrated and performed so far, how will that continue
% \end{itemize}
\subsection{Relevance and quality of additional research training as well as of transferable skills offered}
Within the realization of the  \ksem\ project, the applicant will acquire additional 
scientific skills, such as the new methodologies/computational schemes on top of the 
ones he has also excelled at. Specifically as mentioned above, the outgoing host organization 
is very well experienced in machine learning.  As an example, he has used Support Vector Machines
and Conditional Random Fields for the classification and categorization of household objects.
At Berkeley Mr. Pangercic will deepen his knowledge on reinforcement and apprenticeship learning
to e.g. learn the connection between actions and typical locations or object arrangements.
Furthermore, the applicant will further improve his, during his graduate studies obtained teaching 
skills, by assisting as Teaching Assistant and supervising the undergraduate students.
Due to Prof. Abbeel's excellent connections in the machine learning society the applicant 
will start publishing papers and attend following conferences with machine learning topics:
NIPS, AAAI, ICDL, ICML. In view of the several international collaborations related directly 
or indirectly to this area and of the many short time visitors hosted by both host groups each year, 
the applicant will have the  chance to discuss and present his work to several scientists and 
receive feedback in an international environment. The primary scientific and 
complementary skills described above, expected to be acquired during the \ksem\ 
project, will provide the applicant  with  the necessary stature allowing  him to begin an 
independent research career as academic group leader in one of the leading countries in personal
robotics.
% \begin{itemize}
% \item How will I get better because of Pieter and the rest of the collaborators?
% \item Mention exchange with Bosch.
% \item teaching assistant, 
% \item mention giving a talk
% \item results
% \item attending different conferences
% \end{itemize}
\subsection{Host expertise in training  experienced researchers in the field and capacity to provide 
mentoring/tutoring} 
The host supervision will take place in different levels. Informal discussions in person over
particular scientific or technical issues (implementations of the above listed three 
work packages, understanding the novelty of the approaches, etc) will take place
at Berkeley. It is planned to have regular monthly teleconferences between the applicant, 
and the return and outgoing host organization in addition to regular email
updates. Applicant is also expected to spend up to two weeks and up to four weeks
at the return host organization in first and second year of the proposal respectively.

Prof. Beetz has a great experience in training and mentoring junior researchers, justified 
by the fact that most of his graduate students now hold a faculty positions in academic 
institutions or high-tech companies around the world. To name a few, Dr. Freek Stulp is thus with TU Berlin, 
Dr. Oscar Martinez with Kyushu University, Dr. Alexandra Kirsch holds a junior professor position at 
TU Munich, Dr. Radu Bogdan Rusu and Dr. Suat Gedikli are with WillowGarage, Inc.

Prof. Abbeel is an early career professor and has received multiple awards for his excellent work.
His postdoctoral fellow Jur van den Berg is now an assistant professor at the University of Utah.
His graduate and undergraduate students regularly publish in first-class conferences and the latter
get accepted to the most prestigious graduate programs. Prof. Abbeel meets with his post-docs for 
a couple of hours per week for a personal discussion and has them join some of his meetings with 
the graduate students which he individually supervises. His postdocs also lead reading groups
focused on particular topics. While prof. Abbeel will provide ``hands-on''
active supervision, prof. Goldberg Ken will do a high-level brainstorming with the applicant about 
future research directions, grant proposals, etc. Prof. Goldberg trained and mentored a large number
of junior researchers. prof. Jur van den Berg, University of Utah, prof. Kris Hauser, Indiana University at Bloomington, 
prof. Ron Alterovitz, University of North Carolina at Chapel Hill, prof.  Vladlen Koltun, Stanford University
are just to name a few.

Though the applicant's curriculum gives high expectations that  he will succeed in performing the project 
in quite an independent way, tutoring on particular aspects (like the issues mentioned above) not yet 
familiar to the applicant will be provided.
% \begin{itemize}
% \item Continuation of tutoring, assistance through ROS, teaching, meetings, conference workshop, reports, reading groups, 
% \item workflow, past people that he has trained, 
% \item outgoing host
% \item return host
% \item teaching transferable skills
% \end{itemize}
\newpage


