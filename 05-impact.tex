\section{IMPACT} %4 pages
\label{sec:impact}
\subsection{Potential for acquiring competencies during the fellowship to improve the prospects of 
reaching and/or reinforcing a position of professional maturity, diversity and independence, 
in particular through exposure to transferable skills training}
\todo{Eric: convert into text}\\
\begin{itemize}
\item People Management training skills = supervision of people / students / PhD students. 
\item Budget Administration = knowing how to use a budget. Learning by doing with the project budget at Berkeley and at TUM.
\item Intellectual property Rights training = follow courses in US and in Germany.
\item Business acumen. How to create a company = 2 courses with UnternehmerTUM for scientists back at TUM in Year 3.
\item Possibly another People Management course to follow at TUM in Year 3 with the Life-long learning support unit scheme (people management course for University staff).
\item How to write a grant = Course to follow back at TUM.
\end{itemize}
\subsection{Contribution to career development or re-establishment, where relevant}
The benefits of the proposed training period at Berkeley from a mid/long-term perspective will 
be essential. It will provide for continuity in the career of the applicant, and will allow for the consolidation and 
widening of his career prospects. All skills obtained during the \ksem\ project, ranging 
from computational, methodological, communicative, and collaborative together with 
teaching and mentoring, will allow him to achieve professional maturity and to become an 
independent researcher in his area of interest. Not to mention the extremely beneficial 
outcomes of the applicant's stay in a highly advanced and excelling educational  and 
research environment. A stay is expected to boost his academic skills and enhance his career development.

\subsection{Potential for creating long term collaborations and mutually beneficial cooperation between 
Europe and the Other Third Country} 
\todo{Pieter, Ken: Check if OK}\\
As a member of TUM, the applicant  will endeavor to contribute to create long-term
fruitful  collaborations  between  Berkeley and TUM through  a triple strategy 
agreed between the applicant, the IAS  leader Prof.  Beetz and the  TUM Department  
of  Computer Sciences:\\
\emph{Exchanges  of students  up  to  Master level}\\  
Study  Projects will  be planned in common between TUM and  Berkeley
(starting  from  applicant's  former   research  group)  mentors  and  mentored
students.  Short training  periods will  be planned  on both  sides to
execute these  projects, completed  by a common  \textbf{summer school}  in the
second  year  including all  hands-on  workshops, poster  competition,
invited  internationally  well-known   speakers  of  the  field.   TUM
students will benefit  from travel bursaries to be  applied from usual
German  sources  (DAAD,   Alexander  von  Humboldt  Foundation,  Bosch
Foundation, etc). A duty of  the newly recruited staff is to maintain
a small database of potential grant possibilities to support this very
relation. The applicant will be supported  by the TUM-Forte office to identify the
best  funding  sources,  and    will  oversee  the  quality  of  the
applications written by the students.  This shall ensure a pipeline of
regular  financially  supported  student   projects  at  the  core  of
long-term  exchanges  with   Berkeley. Berkeley  students will equally  apply to
their usual funding bodies to finance study travel abroad.  Both TUM and
Berkeley students  will be  requested on
their return  to report on  their experiences abroad to  their mentors
and to the new students (both technologically and concerning the daily
life).   The  existing  TUM  Master  Programme  \textbf{Robotics,  Cognition,
Intelligence}~\footnote{\url{http://www.in.tum.de/fuer-studieninteressierte/master-studiengaenge/robotics-cognition-intelligence.html}}, 
shall  be the  pipeline of exchanges,  and it  will be
steady  enough after  the conclusion  of applicant's  fellowship.  Thank  to applicant's
professional and  personal experience, applicant will be  a successful example
and  a role  model for  young TUM  students /  Ph.D.  students showing
scientific  and mobility  achievements. One  of applicant's  main contributions
will be to raise  the awareness for the program in the  US and to help
in recruiting more US graduate students.\\
\emph{Expected results:}\\
 3 students exchanged between Berkeley and TUM during
the period of  applicant's fellowship; up to 5 students  in the following five
years after the completion of the IOF contract. A summer school during
the  second year  (either in  Munich or  in Berkeley)  with up  to 20
participants. \\  
\emph{Exchanges of  scientists from  Ph.D. student  level to
post-doc and  senior scientists}\\
 The German Research  Foundation (DFG) offers  a multitude  of  programs to  sustain  research exchanges  and
collaboration with the US,  and complements nicely other programs from
foundations  for  the  same  purpose  (DAAD,  Alexander  von  Humboldt
Foundation, Bosch Foundation, etc). During  the period at TUM and Berkeley, 
Applicant will help to connect principal  investigators of both institutions in order
to win  additional resources  to sustain the  flow of  activities with
University  of California in  Berkeley. In  order to  do that,  I will
exploit the  potentials of  the well-organized offices  network at
TUM, which  will help me at  three levels: a) by  the TUM-Forte office
concerning the  non-scientific parts of the projects  to be submitted;
b)  by the  IAS Chair  administration and  the TUM  Computer Sciences
Financial  Department   concerning  the  financial   details  of  such
proposals;  c)  by  the  TUM  Legal Office  (in  particular  regarding
Intellectual Property Rights  - IPR). Applicant's role will  then remain in the
frame of  scientific exchanges developments,  while the administrative
burden  will be taken  by these  three supporting  services.\\
\emph{Expected results:}\\
Up  to 3  Ph.D. students, 1  post-docs and 1  senior scientists
from  each side  performing research  training or  experiments  in the
corresponding   Berkeley   or   TUM laboratories.  

1 common research projects developed in the first year;
1 common research project (eventually with other partners) financed in
the second year. Clarity of IPR.  
Another possibility - however highly competitive  - for  projects developed
in  common will  be to  insert research  teams from  Berkeley  into FP7
European Research proposals, with a request  to the NSF, or the NIH to
finance the US part.   Ideally, high-level scientific publications can
be  acknowledged  as part  of  this  collaboration in  internationally
significant   peer-review   journals.  A   common   patent  with   its
exploitation  plan would  also  show the  success  of this  privileged
scientific pipeline  between University of California  in Berkeley and
TUM.\\
\emph{Cultural  and corporate understanding}\\
The socialization  part is a
key point  of this plan  and cultural awareness  will be part  of each
proposed  project. This cultural  awareness will  be acquired  on both
sides by  ensuring that  students and scientist  have time  during the
exchanges   to  visit  important   historical  areas,   eventually  be
accommodated  by locals  (especially  for the  students), be  proposed
German and English courses. This  cultural knowledge will be linked to
the visit  of main industrial  sites to approach nearer  the different
industrial cultures (for some of the students it will be an additional
opportunity  to  stay   longer  by  identifying  potential  interested
companies to  work for).  The TUM will  contribute with  enlisting the
Berkeley visitors onto its IKOM (job fair) program of visits to industries.

\subsection{Contribution to European excellence and European competitiveness}
Applicant's  research experience  will  bring to  Berkeley and back  to TUM 
a clear scientific  competence. There are few competences of  that kind in 
Europe that, at  the same time, help to preserve  from the ``Brain Drain''.  
With applicant's additional contacts, the applicant will help to
concretize  other projects  ideas of  the  IAS group and  of the  Computer
Science Faculty  to   answer  the   interests  of  European   companies  (main
partners in this field are: Kuka Roboter GmbH, The Source
Works, Aldebaran  Robotics, Robert Bosch LLC Germany) and  American ones such as 
Willow Garage and Robert Bosch LLC USA.

\subsection{Benefit of the mobility to the European Research Area} 
As  can be  evinced from  applicant's curriculum  vitae, the applicant  
has  a significant mobility  ability  and  wishes  not  only to  maintain  his  scientific
collaborations  with a  recognized world-leading  research institution
(TUM, Berkeley, UTokio, Stanford, Robert Bosch LLC in Palo
Alto etc.), but also  to use the  scientific and  technological links
that he has in e.g. whole ROS community. For the applicant, it will  be 
a thriving scientific time to work in  the USA and back in Munich as  
the Munich R\&D capital encompasses:   Applied   Science   (i.e.,  4   Institutes   Fraunhofer
Gesellschaft); Fundamental research  (i.e., 12 Max-Planck Institutes);
Aerospace \&  IT (i.e., 8 DLR  Institutes including a new  major one in
Robotics;   Health  and   Environment  (i.e.,   23   Helmholtz  Center
Institutes); Corporate R\&D  (about 20500 employees); IT infrastructure
(i.e., Leibniz  Supercomputing Centre); IT services for  more than 100
000 university  customers; and  Competence Center for  Networks (e.g.,
GEANT2, X-WIN,  etc.).  

The excellent  quality of life in  Munich, its
dense and very  dynamic industrial base and the  excellent airport and
train connections will support  my mobility plans (professional duties
as well as  personal journeys) so that I will in  a best position. The
dominant scientific  language of the IAS group  is English, including
with  administrative staff  that can  make  the link  to the  external
administrations (and with the support  of TUM Dual Career office). The
TUM  has also  strong life-long  learning training  offers (technology
transfer entrepreneurship, business acumen).   The applicant is convinced that this
extraordinary  environment   will  be  beneficial   to  strengthen  his
knowledge, to reinforce his collaboration network, and will allow him to
experience the excitement of working in a new country.
\subsection{Impact of the proposed outreach activities}
\todo{Dejan}\\
Demonstration at Berkeley and in Munich.
\newpage