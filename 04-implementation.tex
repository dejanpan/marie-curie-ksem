\section{IMPLEMENTATION} % (MAXIMUM 6 PAGES) 
\label{sec:implementation}
\subsection{Quality of infrastructure / facilities and international collaboration of host} 
\todo{Berkeley}\\
\textbf{TUM}\\
In  recent years, the  Technische Universit\"at  M\"unchen (TUM)  has been
consistently ranked the top academic institution in Germany in several
independent rankings. It provides an excellent environment (likely the
best  in  Germany) by  substantial  funding  from  the Bavarian  state
government,   the  German  government   and  many   private  companies
alike.  One   of  the   missions  of  the   university  is   to  boost
interdisciplinary  research   across  engineering,  natural  sciences,
medicine, and  humanities. Today the  TUM comprises 13  faculties with
more than 23,000  students (about 20 percent of  whom come from abroad
shows the degree of internationalization), 420 professors, and roughly
6,500 academic  and non-academic staff. The TUM is thus well
positioned  to create  new knowledge  and know-how  in Europe  and the
world.  In  2005  the  federal  and  state  governments  started  the
so-called Excellence Initiative in Germany. Between 2006 and 2011 they
will  fund the expansion  of top  university research  with up  to 1.9
billion Euros in three  funding categories: graduate schools, clusters
of excellence  and institutional strategies for  universities. The TUM
was recognized as  one of the first three  universities that succeeded
in   all   three  categories.    The   corporate   concept  ``TUM   The
Entrepreneurial  University'' supports  and advances  the  existing TUM
strategy and  promotes top-level research on multiple  levels. In this
context Entrepreneurial  Spirit means  to activate the  diversity of
human talent  in a concerted,  interactive way. In terms  of top-level
research, it entails combining a  maximum of individual freedom with a
supportive administration.

The proposed returning host  is the TUM Intelligent Autonomous Systems
Group (IAS~\footnote{http://ias.cs.tum.edu/}) from Professor Michael  
Beetz in the Munich city centre, and its extension of the Munich Garching 
Campus. The Chair consists of two  full-time   professors,  1  Junior  
Research   Group  Leader,  15 Ph.D.  students, 2  secretaries.  This  very dynamic  Chair challenges
successfully  seven  main  research  topics: Perception  for  Robots;
Knowledge Processing; Plan-based Control; Cognitive Manipulation; 
Perception of  Human Activities; Facial Expression Recognition.
Its Ph.D.  students co-supervise student projects and  nurture a solid
number of workshops  and seminars in German and  English, ranging from
ROS technologies, Knowledge Representation for Autonomous Robots (IROS
2011   Workshop  for   example),   to  3D   Perception  in   Robotics,
Cognition-enabled Mobile Manipulation  and up to Probabilistic Methods
for Perceiving, Learning and Reasoning about Everyday Activities~\footnote{http://ias.cs.tum.edu/events}. 
The Chair has  extended contacts  with the industry,  as evidenced  by the
number of student  placements in companies in Munich  and elsewhere in
Bavaria, and the ongoing research collaborations with major players in
the domain. In particular  in the robotics field,  with companies
such  as Kuka  GmbH, Aldebaran  Robotics, Willow Garage, Robert Bosch LLC  
and others.   The solid  and excellent  student basis  allows  this Chair  
to  further develop  its scientific path and  will give the applicant upon 
his return  to TUM the possibility of enriching his skills (in research, 
technology transfer, teaching and training).  In  this  context,   the applicant  
will  benefit  from  the  existing infrastructure  to  deliver thriving  
seminar  courses  to a  talented
student body. The  recently established international CoTeSys Graduate
School  (2010) will  extend  its activities  with  the development  of
specific  additional  measures  to  qualify  the  next  generation  of
Ph.D. students, which  the applicant will will contribute to supervise  and train.  The
returning  host  IAS has  extended  research academic  collaborations
abroad (USA, Japan, South-Korea, Europe),  which is a key asset in the
context of this  IOF proposal, as the applicant will be  called upon to contribute
to the reinforcement of scientific  and student exchanges with the USA
and Berkeley University.
Prof.  Beetz,  IAS  Director,   is  additionally  a  member  and  the
vice-coordinator of the CoTeSys  Cluster of Excellence~\footnote{http://www.cotesys.org/} 
coordinated by TUM.  It is  a  close collaboration  between  scientists from  various
disciplines    connecting    neuro-cognitive   and    neuro-biological
foundations to  engineering sciences at  leading research institutions
in  Munich: besides  Technische Universit\"at  M\"unchen,  scientists from
Ludwig-Maximilians-Universit\"at   M\"unchen   -   LMU,  Universit\"at   der
Bundeswehr, Max-Planck Institute  of Neurobiology and German Aerospace
Agency  DLR are involved. CoTeSys
investigates cognition for technical systems such as vehicles, robots,
and   factories.  Cognitive  technical   systems  are   equipped  with
artificial  sensors  and   actuators,  integrated  and  embedded  into
physical  systems, and  act in  the physical  world. They  differ from
other technical systems since  they perform cognitive control and have
cognitive   capabilities.   By   cognitive  capabilities   they   mean
information processing that take into accounts: perception, attention,
memory,  action, learning,  and planning.  The  aim of  CoTeSys is  to
produce future  innovations.  This will  represent another opportunity
for applicant to diversify his activities and find additional partners for R\&D
projects  (whether academic  and local,  or industrial  and  spread in
Europe).
\subsection{Practical arrangements for the implementation and 
management of the research project } 
\todo{Berkeley}\\
\textbf{TUM}\\
The applicant will be inserted in the robust management structure of the IAS. 
Applicant's direct connection with Prof. Michael Beetz, enriched by his interaction 
and supervision of undergraduate and graduate students  will ensure the success of the project. \\
\emph{Grant implementation:}\\
Eric Bourguignon  (CoTeSys Proposal Manager), who  has been associated
to the  non-scientific writing of  this proposal, will deliver  a
set of Marie-Curie FP7  tools to manage the  project, based on
his long-years experience  of EU grants management in  the FP6 and FP7
programs. The tools to be developed specifically for the projects will
include the  basic synopsis of  all contractual issues related  to the
project,  the  yearly reporting  ready-to-use  templates (science  and
finance) and regular follow-up meetings  with the applicant and the Chair 
team to ensure  a quality  delivery of  the expected  results.  The management
style will aim  at letting as much autonomy  in the budget controlling
to the applicant as possible. Both  administrative staff members, Mrs. Walter and
Mrs. Wagner, will take their load  to ensure the smooth running of the
project.

\emph{IAS investments on this project:}
It is estimated  that Prof. Beetz will spend  1,5 person-month on this
project on the 36 months of  the project duration. The IAS will lead
regular  reviewing of the  project and  make first  sure that  the new
fellow integrate into the existing Chair structure. Applicant's re-integration
in  year  3  will  be  facilitated  by my  current  knowledge  of  the
University  structure,  its people,  its  finance,  its other  support
services.

% From their  arrival, through the search of  national and international
% funding programs to the development and the promotion of new projects,
% the recruited  TUM scientists receive  high quality services  from the
% University,  e.g.   from  TUM  Office  for   Research  and  Innovation
% (TUM-Forte),  the  Welcome Office,  the  TUM Corporate  Communications
% Centre and  the TUM Patent and  License Office. Applicant knows  them, 
% and will re-learn quickly how to interact with them.

A strong involvement from the applicant's side will be expected in the design
and  monitoring  organization   of  main  national  and  international
workshops  (\todo{complete}). The applicant  will  coordinate  teams of
students  to  ensure  that  these  events are  successful.   From  the
beginning,  the applicant will fully  dispose his  travel budget  to attend
international  high-level conferences  as well  as to  develop  his own
scientific  network  and scientific  partnerships.   The applicant will  supervise
graduate and undergraduate students.

All details  have been agreed in  common with Prof. Goldberg and Prof. Abbeel  
at Berkeley and Prof. Beetz at TUM, during the preparation of this project proposal.


\subsection{Feasibility and credibility of the project, including work plan} 
In Sections~\ref{sec:q1} and~\ref{sec:q2} we detailed the three tasks to be carried 
out in this project, including the research methodology and the expected results. 
In the figure~\todo{make table or figure} we have represented the work plan including the three 
work packages, plus an additional extra task for the fellowship's training, and we assigned 
an approximate time schedule for each of them. \\
\todo{table, figure}\\

%work plan, gantt diagram, \\
%todo{Eric: Do we need risk analysis???}
\subsection{Practical and administrative arrangements and support for the hosting of the fellow 
} 
\todo{Pieter: revise}
A concrete help for all practical and administrative issues with local authorities related to 
the hosting of foreign researchers is given by the administration core of Berkeley and the 
administrative assistant of the host group. The administration of the partner group will 
provide the applicant with an office space in the host group area, as already mentioned, 
and a Desktop Computer at his disposal with internet access also to the computer 
resources mentioned above. Running costs and consumables which will be needed, like 
postage, copier and printer paper, printer cartridges, pens, scientific software, etc. will 
also be provided by the partner group assistant. 
In addition to the scientific issues mentioned in the previous sections, the administrative 
and financial management of the research fellowship will be provided by the institution's 
administration core and the secretary of the host group. Administration will also help in 
other important and minor issues, such as social security and health insurance, travelling, 
social and cultural activities, etc. Further help in all these issues will be provided directly 
by the supervisor and other members of the partner group.

At TUM, as mentioned earlier, Eric Bourguignon  will assist with all the
Marie Curie FP7 related matters. Both  administrative staff members, Mrs. Walter and
Mrs. Wagner, will take their administrative load  to ensure the smooth running of the
project.
\newpage
